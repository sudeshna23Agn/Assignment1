\newpage

\section{ETHERNET TRANSMITTER}
\subsection{Requirements}

\begin{enumerate}
	\item Receives sequences commands from sequencer and forwards it to Engine computer and CRIO.
	\item Receives Gimbal commands from control and shall forward them to the engine computer.
	\item Receives health status from fault monitor and forwards to engine computer and CRIO.
	\item Receives channel through DAQ channel and sends it to CRIO and telemetry computer.
\end{enumerate}

\subsection{Functional Flow}

\begin{enumerate}
	\item Then Ethernet transmitter shall start and initialize all the IPC message queues.
	\item Upon command from sequencer,the Ethernet transmitter shall initiate TCP connection to CRIO/Telemetry which acts like TCP server.
	\item When the TCP connection is established,the ethernet transmitter packs the received data and sends it to the TCP server.
	
\end{enumerate}



\newpage

\section{TIMER}
\subsection{Requirements}

\begin{enumerate}
  \item The Timer process shall connect to ATS to receive CDT\_HOLD and CDT\_RELEASE commands.
  \item When a CDT\_HOLD command is received,the timer shall signal other processes that hold is issued.
  \item When a CDT\_RELEASE command is received the timer shall signal other processes the release is issued with the updated launch time.
  \item The launch time(TO) shall be issued as absolute epoch time in microseconds and the CDT shall be restarted accordingly.
\end{enumerate}

\subsection{Functional Flow}

\begin{enumerate}
	\item 
	
\end{enumerate}


\newpage

\section{SEQUENCER}
\subsection{Requirements}

\begin{enumerate}
	\item Gets sequences and commands from CRIO and executes it.
	\item Receives the sequence feedback and proceeds next sequence.
	\item The sequencer executes commands in its list,across the CDT time
	\item The commands executed by the flight computer shall be executed within the flight computer.
	\item The commands executed by the EC or CRIO shall be marked as active and 

\end{enumerate}

\subsection{Functional Flow}

\begin{enumerate}
	\item 
	
\end{enumerate}


\newpage

\section{ETHERNET RECEIVER}
\subsection{Requirements}

\begin{enumerate}
	\item Sequences would be forwarded to sequencer.
	\item Health status from CRIO shall be forworded to fault moniter.
	\item if connection from CRIO is gone(fault ID is sent to fault monitor).
	\item Health status from EC shall to forwarded to fault monitor
	\item Sequence feedback shall be forworded to sequencer.
	\item Fault IDs shall be forwarded to fault monitor.
	
\end{enumerate}

\subsection{Functional Flow}

\begin{enumerate}
	\item
	
\end{enumerate}

\subsection{Process states}

\begin{table}[h!]
	\begin{center}
		\begin{tabular}{|p{3 cm}|p{5 cm}|p{5 cm}|}
			\hline
			\textbf{State} & \textbf{Mapped Enum} \\
			\hline
			&  \\
			\hline
			&  \\
			\hline
			&  \\
			\hline
			&  \\
			\hline
			&  \\
			\hline
			&  \\
			\hline
		\end{tabular}
		\begin{center}
			\textbf{Table 1: List of process states that are logged during runtime in Ethernet Receiver }
		\end{center}
	\end{center}
\end{table}

\subsection{List of faults}
\begin{table}[h!]
	\begin{center}
		\begin{tabular}{|p{6 cm}|p{8 cm}|}
			\hline
			\textbf{Fault Group} & \textbf{Fault Description}\\
			\hline
			&   \\
			\hline
			&  \\
			\hline
			&  \\
			\hline
			&  \\
			\hline
			
		\end{tabular}
		\begin{center}
			\textbf{Table 2: List of faults that can occur in Ethernet Receiver}
		\end{center}
	\end{center}
\end{table}
\newpage

\section{GUIDANCE}
\subsection{Requirements}

\begin{enumerate}
	\item 
	\item 
	\item 
	\item 
	
\end{enumerate}

\subsection{Functional Flow}

\begin{enumerate}
	\item
	
\end{enumerate}

\subsection{Process states}

\begin{table}[H]
	\begin{center}
		\begin{tabular}{|p{3 cm}|p{5 cm}|p{5 cm}|}
			\hline
			\textbf{State} & \textbf{Mapped Enum} \\
			\hline
			&  \\
			\hline
			&  \\
			\hline
			&  \\
			\hline
			&  \\
			\hline
			&  \\
			\hline
			&  \\
			\hline
		\end{tabular}
		\begin{center}
			\textbf{Table 1: List of process states that are logged during runtime in }
		\end{center}
	\end{center}
\end{table}

\subsection{List of faults}
\begin{table}[H]
	\begin{center}
		\begin{tabular}{|p{6 cm}|p{8 cm}|}
			\hline
			\textbf{Fault Group} & \textbf{Fault Description}\\
			\hline
			&   \\
			\hline
			&  \\
			\hline
			&  \\
			\hline
			&  \\
			\hline
			
		\end{tabular}
		\begin{center}
			\textbf{Table 2: List of faults that can occur in }
		\end{center}
	\end{center}
\end{table}
\newpage

\section{NAVIGATION}
\subsection{Requirements}

\begin{enumerate}
	\item 
	\item 
	\item 
	\item 
	
\end{enumerate}

\subsection{Functional Flow}

\begin{enumerate}
	\item
	
\end{enumerate}

\subsection{Process states}

\begin{table}[H]
	\begin{center}
		\begin{tabular}{|p{3 cm}|p{5 cm}|p{5 cm}|}
			\hline
			\textbf{State} & \textbf{Mapped Enum} \\
			\hline
			&  \\
			\hline
			&  \\
			\hline
			&  \\
			\hline
			&  \\
			\hline
			&  \\
			\hline
			&  \\
			\hline
		\end{tabular}
		\begin{center}
			\textbf{Table 1: List of process states that are logged during runtime in }
		\end{center}
	\end{center}
\end{table}

\subsection{List of faults}
\begin{table}[H]
	\begin{center}
		\begin{tabular}{|p{6 cm}|p{8 cm}|}
			\hline
			\textbf{Fault Group} & \textbf{Fault Description}\\
			\hline
			&   \\
			\hline
			&  \\
			\hline
			&  \\
			\hline
			&  \\
			\hline
			
		\end{tabular}
		\begin{center}
			\textbf{Table 2: List of faults that can occur in }
		\end{center}
	\end{center}
\end{table}
\newpage

\section{IMU}
\subsection{Requirements}

\begin{enumerate}
	\item 
	\item 
	\item 
	\item 
	
\end{enumerate}

\subsection{Functional Flow}

\begin{enumerate}
	\item
	
\end{enumerate}

\subsection{Process states}

\begin{table}[H]
	\begin{center}
		\begin{tabular}{|p{3 cm}|p{5 cm}|p{5 cm}|}
			\hline
			\textbf{State} & \textbf{Mapped Enum} \\
			\hline
			&  \\
			\hline
			&  \\
			\hline
			&  \\
			\hline
			&  \\
			\hline
			&  \\
			\hline
			&  \\
			\hline
		\end{tabular}
		\begin{center}
			\textbf{Table 1: List of process states that are logged during runtime in }
		\end{center}
	\end{center}
\end{table}

\subsection{List of faults}
\begin{table}[H]
	\begin{center}
		\begin{tabular}{|p{6 cm}|p{8 cm}|}
			\hline
			\textbf{Fault Group} & \textbf{Fault Description}\\
			\hline
			&   \\
			\hline
			&  \\
			\hline
			&  \\
			\hline
			&  \\
			\hline
			
		\end{tabular}
		\begin{center}
			\textbf{Table 2: List of faults that can occur in }
		\end{center}
	\end{center}
\end{table}
